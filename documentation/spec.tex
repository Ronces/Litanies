\documentclass[12pt,a4paper]{article}
\usepackage[utf8x]{inputenc}
\usepackage[french]{babel}
\usepackage{listings}
\usepackage{xcolor}
\usepackage{array}
\usepackage{fancybox}
\usepackage{xspace}
\usepackage{hyperref}
\usepackage{color} 
\usepackage[T1]{fontenc}

\hypersetup{colorlinks=true,urlcolor=blue}

\definecolor{grey}{rgb}{0.95,0.95,0.95} 
\definecolor{green}{rgb}{0.11,0.47,0.11}

\title{Specifications pour la refonte des Litanies de Sang. V0.1}

\begin{document}

\maketitle

\section{Enjeux}

\begin{itemize}
\item créer un site meilleur que le précédent :
\item bugué
\item ne permet pas de créer correctement un personnage
\item lent
\item pas normalisé (complique la fonction de recherche) 
\end{itemize}

\section{Concept/contenu}


\paragraph{On propose un site avec 3 aspects principaux :}
\begin{itemize}
  \item Un site de contenu enrichissable et modifiable par les users (après validation par les admins)
  \item Un module de création de fiches de persos
  \item Un espace personnel facultatif avec conservation de documents relatifs à WoD, et privatisation/partage de ces documents.
\end{itemize}

\subsection{Must have}
\paragraph{Disciplines etc.}
\paragraph{Gestionnaire de personnage: } création de feuilles de perso, calcul de PEX, gestion de lignée, liste de noms par époque etc.
\paragraph{VF}

\subsection{Should have}
\paragraph{Espace perso pour les MJ :} ajout de disciplines, PNJ, PJ, compétences crées par le MJ...
\paragraph{VO}
\paragraph{Outils de calcul de PEX} en fonction de l’historique du vampire
\paragraph{Outils de calcul des périodes de torpeur}
\paragraph{Outils de calcule d’échelle de puissance} qui gagne en combat en fonction de leurs fiches ?
\paragraph{partage de documents} fiches de persos, disciplines personnelles, prophéties in game... entre users/fonction privatisation des documents

\subsection{Nice to have}
\paragraph{réseau social oWoD}


\section{Organisation du site}

\subsection{Création de PJ/PNJ}

\paragraph{V1 :} tableur interactif
des fiches modifiables, copiable

\paragraph{V2 :} design “feuille de perso”, avec un design intégrant les PEX totaux, restants etc.

\paragraph{V3 :} téléchargeable en PDF ou .json
passage vers une version imprimable complète et esthétique



\subsection{Univers}

Un univers est un contexte rassemblant des PJ/PNJ et un ensemble de règles.
Par défault, l’univers est basé sur la version du site (version communément admise des disciplines, rituels etc.).
Le MJ peut paramétrer cet univers en y intégrant des modifications personnelles. Exemple : une version améliorée de Temporis, qu’il pourra appeler Temporis 2.


\subsection{Espace personnel}

\subsubsection{Profil}

\subsubsection{Mes règles}
\begin{itemize}
  \item modifications de l’univers officiel, qui seront appliquées à toutes mes campagnes :
  \item disciplines officielles modifiées
  \item PNJ officiel modifié
  \item création de PNJ ou disciplines applicables à tous les univers
\end{itemize}
\end{document}
